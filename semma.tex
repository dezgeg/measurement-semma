\documentclass[conference,a4paper]{IEEEtran}
\usepackage{lmodern}
\usepackage{microtype}
\usepackage{amsfonts,amsmath,amssymb,amsthm,booktabs,color,enumitem,graphicx,listings,float,tabu}
\usepackage[pdftex,hidelinks]{hyperref}

% correct bad hyphenation here
\hyphenation{op-tical net-works semi-conduc-tor}

\begin{document}
%
% paper title
% Titles are generally capitalized except for words such as a, an, and, as,
% at, but, by, for, in, nor, of, on, or, the, to and up, which are usually
% not capitalized unless they are the first or last word of the title.
% Linebreaks \\ can be used within to get better formatting as desired.
% Do not put math or special symbols in the title.
\title{Network File System Measurements}


% author names and affiliations
% use a multiple column layout for up to three different
% affiliations
\author{\IEEEauthorblockN{Tuomas Tynkkynen}
\IEEEauthorblockA{University of Helsinki\\Department of Computer Science\\00560 Helsinki, Finland\\
Email: tuomas.tynkkynen@iki.fi}
}

% make the title area
\maketitle

% As a general rule, do not put math, special symbols or citations
% in the abstract
\begin{abstract}
Network file systems allow users to access and work on their files over the network,
just as if they were local files.
They are used in universities and large corporations for example,
with simplified backups and being able to use shared workstations as some benefits for them.
In this paper we will examine two popular network file system protocols, NFS and SMB/CIFS.
We will take a look their capabilities and history,
followed by some approaches of measuring the protocols or their implementations.
As good performance has been an important design goal of the protocols,
performance measurements of network file system protocols are well studied.
Specifically, we will investigate Postmark, a generic file system benchmark,
LADDIS, a NFS protocol level benchmark and TODO something else.
Then we will look into measuring performance improvements of NFS protocol version 4.1.
% TODO: the NetApp stuff here as well
\end{abstract}

\section{Introduction to network filesystems}

Network file systems allow files on a server to be shared and accessed over the network.
Several different protocols for network file systems have been designed over the years,
with the most well-known being Network File System (NFS) and Server Message Block (SMB).
Though history and specifics of those protocols differ considerably, the common design
goal shared by both of them is to be transparent to the applications using them.
That is, applications and libraries should work identically on networked file systems
as they work on local file systems, without requiring any code changes or recompilation.

The NFS protocol originates from Sun Microsystems, where the initial implementation work for the
Unix 4.2 operating system was started in 1984~\cite{NFS}. Its original design goals were
transparent operation with existing programs by maintaining existing Unix file system
semantics, ability to recover from server reboots or crashes, and having reasonable
performance. Since then, extensions to the original protocol have been standardized
in several RFCs, with NFSv3~\cite{NFSv3RFC} in 1995 and NFSv4~\cite{NFSv4RFC},
bringing several improvements in performance, security and feature set~\cite{NFSv4Better}.



% trigger a \newpage just before the given reference
% number - used to balance the columns on the last page
% adjust value as needed - may need to be readjusted if
% the document is modified later
%\IEEEtriggeratref{8}
% The "triggered" command can be changed if desired:
%\IEEEtriggercmd{\enlargethispage{-5in}}

% references section

% can use a bibliography generated by BibTeX as a .bbl file
% BibTeX documentation can be easily obtained at:
% http://mirror.ctan.org/biblio/bibtex/contrib/doc/
% The IEEEtran BibTeX style support page is at:
% http://www.michaelshell.org/tex/ieeetran/bibtex/
\bibliographystyle{IEEEtran}
% argument is your BibTeX string definitions and bibliography database(s)
\bibliography{IEEEabrv,viitteet}

% that's all folks
\end{document}
