\section{Introduction to network filesystems}

Network file systems allow files on a server to be shared and accessed over the network.
Several different protocols for network file systems have been designed over the years,
with the most well-known being Network File System (NFS) and 
Server Message Block (SMB).
% Common Internet File System (CIFS).
Though history and specifics of those protocols differ considerably, the common design
goal shared by both of them is to be transparent to the applications using them.
That is, applications and libraries should work identically on networked file systems
as they work on local file systems, without requiring any code changes or recompilation.

The NFS protocol originates from Sun Microsystems, where the initial implementation work for the
Unix 4.2 operating system was started in 1984~\cite{NFS}. Its original design goals were
transparent operation with existing programs by maintaining existing Unix file system
semantics, ability to recover from server reboots or crashes, and having reasonable
performance. Since then, extensions to the original protocol have been standardized
in several RFCs, with NFSv3~\cite{NFSv3RFC} in 1995 and NFSv4~\cite{NFSv4RFC},
bringing several improvements in performance, security and feature set~\cite{NFSv4Better}.

The SMB protocol has a more complex history behind it. The protocol was initially designed
and implemented by IBM for the DOS operating system in the 80s
and ran over NetBIOS on the session layer and NBF (NetBIOS frames) on the network and transfer layers~\cite{CifsBook}.
Further developments to the protocol were made by Microsoft.
In 1992 parts of the the protocol were specified in an X/Open standard~\cite{XopenSmbSpec}.
In 1996, Microsoft started using the Common Internet File System (CIFS) name for the protocol.

\section{Performance measurements}

As mentioned earlier, good performance was one of the design goals of the NFS protocol.
As a result, various benchmarks for NFS server implementations have been implemented.
One instance of such a benchmark is LADDIS from the 90s~\cite{LADDIS}, designed and named after by
several NFS server vendors at that time (\underline{L}egato Systems, \underline{A}uspex Systems, \underline{D}ata General,
\underline{D}igital Equipment, \underline{I}nterphase and \underline{S}un Microsystems).
It is designed to be a standard benchmark for making performance comparisons across different NFS server implementations
or configurations.
The LADDIS benchmarking software is a synthetic workload generator that talks to the NFS server directly,
without any dependency on the NFS client implementation of the system generating the workload.
The implementation of the benchmark consists of making randomly selected NFS protocol operations
(for instance, creating deleting or reading from a file)
to the server.
The distribution of which operations to perform
