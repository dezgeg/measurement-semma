\section{Introduction to network filesystems}

Network file systems allow files on a server to be shared and accessed over the network.
Several different protocols for network file systems have been designed over the years,
with the most well-known being Network File System (NFS) and Server Message Block (SMB).
Though history and specifics of those protocols differ considerably, the common design
goal shared by both of them is to be transparent to the applications using them.
That is, applications and libraries should work identically on networked file systems
as they work on local file systems, without requiring any code changes or recompilation.

The NFS protocol originates from Sun Microsystems, where the initial implementation work for the
Unix 4.2 operating system was started in 1984~\cite{NFS}. Its original design goals were
transparent operation with existing programs by maintaining existing Unix file system
semantics, ability to recover from server reboots or crashes, and having reasonable
performance. Since then, extensions to the original protocol have been standardized
in several RFCs, with NFSv3~\cite{NFSv3RFC} in 1995 and NFSv4~\cite{NFSv4RFC},
bringing several improvements in performance, security and feature set~\cite{NFSv4Better}.

